\documentclass[11pt, letterpaper, notitlepage]{article} 
\usepackage[
    top=1in,
    left=0.75in,
    right=0.75in,
    bottom=1in,
    headheight=60pt,
    % showframe
]{geometry}
\usepackage[T1]{fontenc}
\usepackage{tgadventor}
\usepackage{titlesec}
\usepackage{fancyhdr}
\usepackage{lastpage}
\usepackage{tocloft}
\usepackage[fleqn]{amsmath}
\usepackage{amssymb}
\usepackage{multirow}
% \usepackage{float}
\usepackage{lscape} % For certain tables to be in landscape mode
% \usepackage{etoolbox}
\usepackage{enumitem}

\DeclareMathVersion{sans}
\SetSymbolFont{operators}{sans}{OT1}{cmbr}{m}{n}
\SetSymbolFont{letters}{sans}{OML}{cmbrm}{m}{it}
\SetSymbolFont{symbols}{sans}{OMS}{cmbrs}{m}{n}
\SetMathAlphabet{\mathit}{sans}{OT1}{cmbr}{m}{sl}
\SetMathAlphabet{\mathbf}{sans}{OT1}{cmbr}{bx}{n}
\SetMathAlphabet{\mathtt}{sans}{OT1}{cmtl}{m}{n}
\SetSymbolFont{largesymbols}{sans}{OMX}{iwona}{m}{n}

\title{MEEN 621 Notes}
\author{Shivanand P}

\titleformat{\section}[block]{\bfseries \sffamily \large}{\thesection.}{5pt}{}
\titleformat{\subsection}[block]{\bfseries \sffamily \normalsize}{\thesubsection.}{5pt}{}
\titleformat{\subsubsection}[block]{\bfseries \sffamily \small}{}{0pt}{}

\makeatletter
% \patchcmd{\LS@rot}{90}{-90}{}{}
% \patchcmd{\endlandscape}{90}{-90}{}{}
\newenvironment{mywideralign*}{\ifvmode\else\hfil\null\linebreak\fi
  \hspace*{-\leftmargin}\minipage\textwidth
  \setlength{\abovedisplayskip}{0pt}%
  \setlength{\abovedisplayshortskip}{\abovedisplayskip}%
  \start@align\@ne\st@rredtrue\m@ne}%
{\endalign\endminipage\linebreak}
\makeatother
\begin{document}

\sffamily
\sffamily\mathversion{sans}

\maketitle
\renewcommand{\cftpartleader}{\cftdotfill{\cftdotsep}} % for parts
\renewcommand{\cftsecleader}{\cftdotfill{\cftdotsep}}
\tableofcontents
\thispagestyle{empty}

\newpage
\setcounter{page}{1}

\pagestyle{fancy}
\fancyhead{}    
\fancyhead[R]{\normalfont\sffamily Shivanand P \\ MEEN 621 Notes}
\fancyfoot{}
\fancyfoot[R]{\normalfont\sffamily Page \textbf{\thepage} of \textbf{\pageref{LastPage}}}
\renewcommand{\headrulewidth}{0pt}
\renewcommand{\footrulewidth}{0pt}
\renewcommand{\arraystretch}{1.75}

\section{Mathematics}

\subsection{First-Order Linear ODE}
Equations of the form $\frac{dy}{dx} + \alpha(x) y = \beta(x)$ can be solved by multiplying both sides by an integrating factor $e^{\int_{}^{}\alpha(x)dx}$. Then the left-hand side becomes $\frac{d(e^{\int_{}^{}\alpha(x)dx}.y)}{dx}$ and the right hand side becomes $e^{\int_{}^{}\alpha(x)dx} \beta(x)$. The left and right hand sides can then be integrated directly.

\section{Continuum Flow}
\subsection{Mass Conservation}
\begin{align*}
\frac{\partial \rho}{\partial t} + \vec{\nabla} . (\rho v_j) &= 0 \\
\frac{\partial \rho}{\partial t} + \frac{\partial (\rho v_j)}{\partial x_j} &= 0 \\
\end{align*}

\subsubsection{Cartesian Coordinates}
\begin{align*}
\frac{\partial \rho}{\partial t} + \frac{\partial (\rho v_1)}{\partial x_1} + \frac{\partial (\rho v_2)}{\partial x_2} + \frac{\partial (\rho v_3)}{\partial x_3} &= 0 \\
\end{align*}

\subsubsection{Cylindrical Polar Coordinates}
\begin{align*}
\frac{\partial \rho}{\partial t} + \frac{1}{r} \frac{\partial}{\partial r} (r \rho v_r) + \frac{1}{r} \frac{\partial \rho v_{\theta}}{\partial \theta} + \frac{\partial \rho v_z}{\partial z} &= 0 \\
\end{align*}

\subsection{Momentum Conservation (Navier-Stokes Equations)}
$\rho \biggl[\dfrac{\partial{\vec{v}}}{\partial t} + (\vec{v}.\vec{\nabla})\vec{v}\biggr] = \rho \vec{f} -\vec{\nabla} p + \mu \nabla^2 \vec{v}$

\subsubsection{Cartesian Coordinates}
\begin{align*}
\rho \biggl[ \frac{\partial v_1}{\partial t} + v_1 \frac{\partial v_1}{\partial x_1} + v_2 \frac{\partial v_1}{\partial x_2} + v_3 \frac{\partial v_1}{\partial x_3} \biggr] &= \rho f_1 - \frac{\partial p}{\partial x_1} + \mu \biggl[ \frac{\partial^2 v_1}{\partial x^2_1} + \frac{\partial^2 v_1}{\partial x^2_2} + \frac{\partial^2 v_1}{\partial x^2_3} \biggr] \\
\rho \biggl[ \frac{\partial v_2}{\partial t} + v_1 \frac{\partial v_2}{\partial x_1} + v_2 \frac{\partial v_2}{\partial x_2} + v_3 \frac{\partial v_2}{\partial x_3} \biggr] &= \rho f_2 - \frac{\partial p}{\partial x_2} + \mu \biggl[ \frac{\partial^2 v_2}{\partial x^2_1} + \frac{\partial^2 v_2}{\partial x^2_2} + \frac{\partial^2 v_2}{\partial x^2_3} \biggr] \\
\rho \biggl[ \frac{\partial v_3}{\partial t} + v_1 \frac{\partial v_3}{\partial x_1} + v_2 \frac{\partial v_3}{\partial x_2} + v_3 \frac{\partial v_3}{\partial x_3} \biggr] &= \rho f_3 - \frac{\partial p}{\partial x_3} + \mu \biggl[ \frac{\partial^2 v_3}{\partial x^2_1} + \frac{\partial^2 v_3}{\partial x^2_2} + \frac{\partial^2 v_3}{\partial x^2_3} \biggr] \\
\end{align*}

\subsubsection{Cylindrical Polar Coordinates}
\begin{mywideralign*}
\rho \biggl[ \frac{\partial v_r}{\partial t} + v_r \frac{\partial v_r}{\partial r} + \frac{v_{\theta}}{r} \frac{\partial v_r}{\partial \theta} + v_z \frac{\partial v_r}{\partial z} - \frac{{v^2}_{\theta}}{r} \biggr] &= \rho f_r - \frac{\partial p}{\partial r} + \mu \biggl[ \frac{\partial}{\partial r} \biggl( \frac{1}{r} \frac{\partial}{\partial r} ( r v_r ) \biggr) + \frac{1}{r^2} \frac{{\partial}^2 v_r}{\partial {\theta}^2} + \frac{{\partial}^2 v_r}{\partial z^2} -\frac{2}{r^2} \frac{\partial v_{\theta}}{\partial \theta}  \biggr] \\
\rho \biggl[ \frac{\partial v_{\theta}}{\partial t} + v_r \frac{\partial v_{\theta}}{\partial r} + \frac{v_{\theta}}{r} \frac{\partial v_{\theta}}{\partial \theta} + v_z \frac{\partial v_{\theta}}{\partial z} + \frac{v_r v_{\theta}}{r} \biggr] &= \rho f_{\theta} - \frac{1}{r} \frac{\partial p}{\partial \theta} + \mu \biggl[ \frac{\partial}{\partial r} \biggl( \frac{1}{r} \frac{\partial}{\partial r} ( r v_{\theta} ) \biggr) + \frac{1}{r^2} \frac{{\partial}^2 v_{\theta}}{\partial {\theta}^2} + \frac{{\partial}^2 v_{\theta}}{\partial z^2} +\frac{2}{r^2} \frac{\partial v_r}{\partial \theta}  \biggr] \\
\rho \biggl[ \frac{\partial v_z}{\partial t} + v_r \frac{\partial v_z}{\partial r} + \frac{v_{\theta}}{r} \frac{\partial v_z}{\partial \theta} + v_z \frac{\partial v_z}{\partial z} \biggr] &= \rho f_z - \frac{\partial p}{\partial z} + \mu \biggl[ \frac{1}{r} \frac{\partial}{\partial r} \biggl( r \frac{\partial v_z}{\partial r} \biggr) + \frac{1}{r^2} \frac{{\partial}^2 v_z}{\partial {\theta}^2} + \frac{{\partial}^2 v_z}{\partial z^2} \biggr]
\end{mywideralign*}
  

\section{Vorticity Dynamics}
\begin{align*}
\text{Circulation } \Gamma &= \oint_{C}^{} \vec{v} . \vec{dl} = \int_{S}^{} \vec{\omega}.\hat{n}\ dA \text{ (Flux of Vorticity across surface } S \text{)}\\ 
\text{Average angular velocity } \bar{\Omega} &= \frac{\bar{u}_{\theta}}{a} = \frac{\oint_{C}^{} \vec{v}.\vec{dl}}{2 \pi a^2} =  \frac{\Gamma}{2 \pi a^2} \text{ (On a circle of radius } a\text{)} = \frac{{\omega}_j n_j}{2}\\
\text{For Irrotational Flow, } \Gamma &= 0,\ \vec{\omega} = 0 \\
\text{A vector field } \vec{\alpha} \text{ is considered solenoidal if } \vec{\nabla} . \vec{\alpha} &= 0 \\
\vec{\nabla}.\vec{\omega} &= \vec{\nabla}.(\vec{\nabla} \times \vec{v}) = 0 \\
\end{align*}

\subsection{Streamlines and Vortex Lines}
\begin{align*}
\text{Equations for Streamlines (Cartesian Coordinates) } \frac{dx}{u_x} &= \frac{dy}{u_y} = \frac{dz}{u_z} \\
\text{Equations for Streamlines (Cylindrical Polar Coordinates) } \frac{dR}{u_R} &= \frac{R d \phi}{u_\phi} = \frac{dz}{u_z} \\
\text{Equations for Vortex Lines (Cartesian Coordinates) } \frac{dx}{\omega_x} &= \frac{dy}{\omega_y} = \frac{dz}{\omega_z} \\
\end{align*}

\subsection{Terms}
\begin{itemize}
\item \textbf{Barotropic}: Density is a function of pressure only. $\dfrac{1}{\rho^2} \vec{\nabla} \rho \times \vec{\nabla} p = 0$ 
\item \textbf{Inviscid}: Viscosity can be neglected. $\nu \nabla^2 \vec{\omega} = 0$ 
\item \textbf{Baroclinic}: Measure of how misaligned the gradient of pressure is from the gradient of density in a fluid $\vec{\nabla} \rho \times \vec{\nabla} p$
\item \textbf{Isobars}: Constant pressure lines
\item \textbf{Isopycals}: Constant density lineses
\end{itemize}

\subsection{Vorticity Transport Equation}
\begin{align*}
\frac{D\vec{\omega}}{Dt} &= (\vec{\omega}.\vec{\nabla}) \vec{v} + \frac{1}{\rho^2} \vec{\nabla} \rho \times \vec{\nabla} p + \nu \nabla^2 \vec{\omega} \\
\frac{\partial \vec{\omega}}{\partial t} + (\vec{v}.\vec{\nabla})\vec{\omega} &= (\vec{\omega}.\vec{\nabla}) \vec{v} + \frac{1}{\rho^2} \vec{\nabla} \rho \times \vec{\nabla} p + \nu \nabla^2 \vec{\omega}
\end{align*}

\subsubsection{Legend}
\begin{itemize}
\item $\dfrac{D\vec{\omega}}{Dt}$: Total rate of change of vorticity of a fluid particle
\item $(\vec{\omega}.\vec{\nabla}) \vec{v}$: Vorticity production due to stretching/tilting of vortex lines
\item $\dfrac{1}{\rho^2} \vec{\nabla} \rho \times \vec{\nabla} p$: Vorticity production due to baroclinic effects
\item $\nu \nabla^2 \vec{\omega}$: Viscous diffusion of vorticity which dissipates or redistributes vorticity
\end{itemize}

\subsection{Helmholtz's Vortex Theorems for Inviscid Flows}
\subsubsection{Assumptions}
\begin{itemize}
\item Inviscid
\item Barotropic
\item Conservative Body Forces $\vec{F} = -\vec{\nabla{\psi}}$
\end{itemize}

\subsubsection{Statement}
\begin{itemize}
\item Fluid particles/element orignially free of vorticity remain free of vorticity (non-rotating)
\item Vortex lines (tubes) move with the fluid for inviscid flows. Vortex line is always comprised of the same fluid particles (Vorticity is frozen to the flow for inviscid flow)
\item The strength of the vortex tube (circulation) does not vary with time during the fluid motion. Vortex tubes must be closed, go to infinity, or end on solid boundaries.
\end{itemize}

\subsection{Kelvin's circulation theorem}
\subsubsection{Assumptions}
\begin{itemize}
\item Inviscid
\item Barotropic
\item Incompressible
\item Conservative Body Forces $\vec{F} = -\vec{\nabla{\psi}}$
\end{itemize}

\subsubsection{Statement}
The circulation (strength of the vortex) around a closed curve moving with the fluid will remain constant.

\begin{align*}
\frac{\partial \Gamma_C}{\partial t} + v_j \frac{\partial \Gamma_C}{\partial x_j} &= \oint_{C}^{} \nu \nabla^2\vec{v}.\vec{dx} \\
\text{For inviscid flow, } \frac{\partial \Gamma_C}{\partial t} + v_j \frac{\partial \Gamma_C}{\partial x_j} &= 0
\end{align*}

\subsection{Bernoulli's Equation}
\subsubsection{Assumptions}
\begin{itemize}
\item Inviscid $\nabla^2 v = 0$ $\rightarrow$ Euler
\item Only gravitational body forces (Conservative) $\rightarrow$ $f_i = -\dfrac{d \psi}{dx_i}$ where $\psi = g z$
\item Barotropic $\rightarrow$ $\rho = \rho(p)$
\item Steady flow $\rightarrow$ $\dfrac{\partial}{\partial t} = 0$ 
\end{itemize}
Note: No restriction on the compressiblity effects

\subsubsection{Statement}
\begin{align*}
\nabla \Biggl[\frac{\vec{v}.\vec{v}}{2} + g z + \int \frac{dp}{\rho} \Biggr] &= -\vec{\omega} \times \vec{v} \\
\text{Along streamline or vortex line } \frac{\vec{v}.\vec{v}}{2} + g z + \int \frac{dp}{\rho} &= \text{constant}\\
\text{Unsteady flow } \frac{\partial \phi}{\partial t} + \frac{\vec{v}.\vec{v}}{2} + g z + \int \frac{dp}{\rho} &= F(t)\\
\text{Steady Potential flow } \frac{\vec{v}.\vec{v}}{2} + g z + \int \frac{dp}{\rho} &= \text{constant throughout the flow}\\
\end{align*}

\section{Potential Flow}
\begin{align*}
i &= \sqrt{-1} \\
z &= x + i y = r e^{i \theta} = r (\cos{\theta} + i \sin{\theta}) \\
\bar{z} &= x - i y = r e^{-i \theta} = r (\cos{\theta} - i \sin{\theta}) \\
x &= r \cos{\theta} \\
y &= r \sin{\theta} \\
r &= \sqrt{x^2 + y^2} = |z| = |z \bar{z}|^{\frac{1}{2}} \\
\theta &= arctan{\frac{y}{x}} \\
\text{2D Incompressible Potential Flow } \nabla^2 \phi &= 0 = \nabla^2 \psi \\
\text{Complex Potential\ } F(z) &= \phi(z) + i \psi(z) \\
\text{Complex Velocity\ } w(z) &= u - i v = (v_r - i v_{\theta}) e^{-i \theta} = \frac{dF(z)}{dz} \\
\bar{w}(z) &= u + i v = (v_r + i v_{\theta}) e^{i \theta} \\
\end{align*}


\subsection{Stream Function $\Leftrightarrow$ Potential Function $\Leftrightarrow$ Velocities}
Also called Cauchy Reimann Equations
\begin{align*}
u &= \frac{\partial \phi}{\partial x} = \frac{\partial \psi}{\partial y} \\
v &= \frac{\partial \phi}{\partial y} = -\frac{\partial \psi}{\partial x} \\ \\
v_r &= \frac{\partial \phi}{\partial r} = \frac{1}{r} \frac{\partial \psi}{\partial \theta} \\
v_{\theta} &= \frac{1}{r} \frac{\partial \phi}{\partial \theta} = -\frac{\partial \psi}{\partial r} \\
\end{align*}

\subsection{Cartesian $\Leftrightarrow$ Polar Velocities}
\begin{align*}
u &= v_r \cos{\theta} - v_{\theta} \sin{\theta} \\
v &= v_r \sin{\theta} + v_{\theta} \cos{\theta} \\ \\
v_r &= u \cos{\theta} + v \sin{\theta} \\
v_{\theta} &= -u \sin{\theta} + v \cos{\theta} \\
\end{align*}

\subsection{Complex Potentials}

\begin{landscape}

\begin{tabular}{|c|c|c|c|c|c|c|}
\hline & F(z) & $\phi$ & $\psi$ & w(z) & $u$ and $v_r$ & $v$ and $v_{\theta}$ \\

\hline \multirow{2}{*}{Uniform}
  & \multirow{2}{*}{$U e^{-i \alpha} z$} % F(z)
    & $U(x \cos{\alpha}+y \sin{\alpha})$ % phi(x,y)
    & $U(y \cos{\alpha}-x \sin{\alpha})$ % psi(x,y)
  & \multirow{2}{*}{$U e^{-i \alpha}$} % w(z)
    & $U \cos{\alpha}$ % u(x,y)
    & $U \sin{\alpha}$ % v(x,y)
  \\ \cline{3-4} \cline{6-7} 
  & 
    & $U r \cos(\theta - \alpha)$ % phi(r,theta)
    & $U r \sin(\theta - \alpha)$ % psi(r,theta)
  & 
    & $U \cos(\theta - \alpha)$ % v_r(r,theta)
    & $-U \sin(\theta - \alpha)$ % v_theta(r,theta)
\\

\hline \multirow{2}{*}{Corner}
  & \multirow{2}{*}{$C z^n$} % F(z)
    & % phi(x,y)
    & % psi(x,y)
  & \multirow{2}{*}{$n C z^{n-1}$} % w(z)
    & % u(x,y)
    & % v(x,y)
  \\ \cline{3-4} \cline{6-7} 
  & 
    & $C r^n \cos{n \theta}$ % phi(r,theta)
    & $C r^n \sin{n \theta}$ % psi(r,theta)
  & 
    & $n C r^{n-1} \cos[(n-1) \theta]$ % v_r(r,theta)
    & $- n C r^{n-1} \sin[(n-1) \theta]$ % v_theta(r,theta)
\\

\hline \multirow{2}{*}{Source/Sink}
  & \multirow{2}{*}{$\frac{m}{2 \pi} \ln(z - z_0)$} % F(z)
    & $\frac{m}{4 \pi} \ln[{x^2+y^2}]$ % phi(x,y)
    & $\frac{m}{2 \pi} \arctan{\frac{y}{x}}$ % psi(x,y)
  & \multirow{2}{*}{$\frac{m}{2 \pi (z - z_0)}$} % w(z)
    & $\frac{m}{2 \pi} \frac{x}{x^2+y^2}$ % u(x,y)
    & $\frac{m}{2 \pi} \frac{y}{x^2+y^2}$ % v(x,y)
  \\ \cline{3-4} \cline{6-7} 
  & 
    & $\frac{m}{2 \pi} \ln{r}$ % phi(r,theta)
    & $\frac{m}{2 \pi} \theta$ % psi(r,theta)
  & 
    & $\frac{m}{2 \pi r}$ % v_r(r,theta)
    & $0$ % v_theta(r,theta)
\\

\hline \multirow{2}{*}{Free Vortex} % Type
  & \multirow{2}{*}{$-\frac{i \Gamma}{2 \pi} \ln(z - z_0)$} % F(z)
    & $\frac{\Gamma}{2 \pi} \arctan{\frac{y}{x}}$ % phi(x,y)
    & $-\frac{\Gamma}{4 \pi} \ln[x^2 + y^2]$ % psi(x,y)
  & \multirow{2}{*}{$-\frac{i \Gamma}{2 \pi (z-z_0)}$} % w(z)
    & $-\frac{\Gamma}{2 \pi} \frac{y}{x^2 + y^2}$ % u(x,y)
    & $\frac{\Gamma}{2 \pi} \frac{x}{x^2 + y^2}$ % v(x,y)
  \\ \cline{3-4} \cline{6-7} 
  & 
    & $\frac{\Gamma}{2 \pi} \theta$ % phi(r,theta)
    & $-\frac{\Gamma}{2 \pi} \ln{r}$ % psi(r,theta)
  & 
    & $0$ % v_r(r,theta)
    & $\frac{\Gamma}{2 \pi r}$ % v_theta(r,theta)
\\

\hline \multirow{2}{*}{Dipole} % Type
  & \multirow{2}{*}{$\dfrac{\mu}{\pi z}$} % F(z)
    & $\frac{\mu}{\pi} \frac{x}{x^2+y^2}$ % phi(x,y)
    & $-\frac{\mu}{\pi} \frac{y}{x^2+y^2}$ % psi(x,y)
  & \multirow{2}{*}{$-\dfrac{\mu}{\pi z^2}$} % w(z)
    & $\frac{\mu}{\pi} \frac{y^2-x^2}{(x^2+y^2)^2}$ % u(x,y)
    & $-\frac{\mu}{\pi} \frac{2xy}{(x^2+y^2)^2}$ % v(x,y)
  \\ \cline{3-4} \cline{6-7} 
  & 
    & $\frac{\mu}{\pi r} \cos{\theta}$ % phi(r,theta)
    & $-\frac{\mu}{\pi r} \sin{\theta}$ % psi(r,theta)
  & 
    & $-\frac{\mu}{\pi r^2} \cos{\theta}$ % v_r(r,theta)
    & $-\frac{\mu}{\pi r^2} \sin{\theta}$ % v_theta(r,theta)
\\

% \hline \multirow{2}{*}{} % Type
%   & \multirow{2}{*}{$$} % F(z)
%     & $$ % phi(x,y)
%     & $$ % psi(x,y)
%   & \multirow{2}{*}{$$} % w(z)
%     & $$ % u(x,y)
%     & $$ % v(x,y)
%   \\ \cline{3-4} \cline{6-7} 
%   & 
%     & $$ % phi(r,theta)
%     & $$ % psi(r,theta)
%   & 
%     & $$ % v_r(r,theta)
%     & $$ % v_theta(r,theta)
% \\

\hline
\end{tabular}
 
\vspace{-4mm}

\subsubsection{Legend}
\setlist{topsep=0pt, leftmargin=*}
\begin{itemize}
  \item Uniform
  \begin{itemize}
    \item $U$: Uniform velocity magnitude
    \item $\alpha$: Angle of attack - the angle at which the direction of the uniform velocity is oriented with respect to the horizontal
  \end{itemize}
  \item Corner
  \begin{itemize}
    \item $C$: Indicates the direction. $C > 0$ always
    \item $n$: $\dfrac{\pi}{\text{angle of the corner}}$
  \end{itemize} 
  \item Source/Sink
  \begin{itemize}
    \item $m$: Volume flow rate per unit dimension normal to the page. For source, $m > 0$, whereas for sink, $m < 0$
  \end{itemize}
  \item Dipole/Doublet
  \begin{itemize}
    \item $a$: Half distance between the source and sink
    \item $Q$: Volume flow rate per unit dimension normal to the page
    \item $\mu$: $Q a$
  \end{itemize}        
\end{itemize}

\end{landscape}

\subsection{Infinite Series Expansions}
\begin{itemize}
  \item $\ln(1+\epsilon) = \epsilon - \frac{\epsilon^2}{2} + \frac{\epsilon^3}{3} - \dots$ when $|\epsilon| \le 1$
  \item $(1+\epsilon)^{-1} = 1 - \epsilon + \epsilon^2 - \epsilon^3 + \dots$ when $|\epsilon| \le 1$
  \item $(1+\epsilon)^{\frac{1}{2}} = 1 + \frac{1}{2} \epsilon - \frac{1}{8} \epsilon^2 + \frac{1}{16} \epsilon^3 - \frac{5}{128} \epsilon^4 + \frac{7}{256} \epsilon^5 - \dots$ when $|\epsilon| \le 1$
  \item $(1+\epsilon)^{-\frac{1}{2}} = 1 - \frac{1}{2} \epsilon + \frac{3}{8} \epsilon^2 - \frac{5}{16} \epsilon^3 + \frac{35}{128} \epsilon^4 - \frac{63}{256} \epsilon^5 + \dots$ when $|\epsilon| \le 1$
  \item $e^{\epsilon} = 1 - \frac{\epsilon^2}{2!} + \frac{\epsilon^4}{4!} - \dots$
  \item $\sin(\epsilon) = \epsilon - \frac{\epsilon^3}{3!} + \frac{\epsilon^5}{5!} - \frac{\epsilon^7}{7!} + \dots$ 
\end{itemize}

\subsection{Bernoulli's Equation (Irrotational)}
$p_{\infty} + \dfrac{\rho v^2_{\infty}}{2} = p + \dfrac{\rho |v^2|}{2} $

\subsection{Forces on a 2D Body}
\textit{All the below forces have the units of Force per unit length normal to the sheet of paper}
\begin{align*}
\text{Drag Force} &= -\int_{C}^{} (p-p_{\infty}) \cos{\theta}\ ds \\
\text{Lift Force} &= -\int_{C}^{} (p-p_{\infty}) \sin{\theta}\ ds \\
\text{Complex Force } G = D - i L &= -i \oint_{C}^{} p d\bar{z} = -i \oint_{C}^{} \Biggl[ p_{\infty} + \dfrac{\rho v^2_{\infty}}{2} -  \frac{\rho |v^2|}{2} \Biggr] d\bar{z} \\
&= \frac{i \rho}{2} \oint_{C}^{} |v^2| d\bar{z} = \frac{i \rho}{2} \oint_{C}^{} w \bar{w} d\bar{z} \\
&= \frac{i \rho}{2} \oint_{C}^{} [w(z)]^2 dz \ \bigl(\text{First Blasius Integral Law}\bigr) \\
\end{align*}

\subsection{Residue Theorem}
\begin{align*}
\text{If } F(z) &= \sum_{j}^{} \frac{B_j}{z - z_j} \\
R_k = \sum B \\
\oint_{C}^{} F(z) dz &= 2 \pi i \sum_{k}^{} R_k \\
\end{align*}

\end{document}